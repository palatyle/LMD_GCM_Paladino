{\footnotesize
\begin{verbatim}
#-----------------------------------------------------------------------
# Run parameters for the rcm1d.e model
#-----------------------------------------------------------------------

#### Time integration parameters
#
# Initial date (in martian sols ; =0 at Ls=0)
day0=0
# Initial local time (in hours, between 0 and 24)
time=0
# Number of time steps per sol
day_step=48
# Number of sols to run
ndt =400

#### Physical parameters
#
# Surface pressure (Pa)
psurf=7000.
# Gravity (ms^-2)
g=3.72
# Molar mass of atmosphere (g)
mugaz=43.49
# Specific heat capacity of atmosphere?
cpp=744.5
# latitude (in degrees)
latitude=0.0
# orbital distance at perihelion (AU)
periastr=1.558
# orbital distance at aphelion (AU)
apoastr=1.558
# obliquity (degrees)
obliquit=0.0
# Solar zenith angle (degrees)
szangle=60.0

# Albedo of bare ground
albedo=0.2
# Emissivity of bare ground
emis=1.0
# Soil thermal inertia (SI)
inertia=400
# zonal eastward component of the geostrophic wind (m/s)
u=10.
# meridional northward component of the geostrophic wind (m/s)
v=0.
# Initial CO2 ice on the surface (kg.m-2)
co2ice=0
# hybrid vertical coordinate ? (.true. for hybrid and .false. for sigma levels)
hybrid=.false.
# autocompute vertical discretisation? (useful for exoplanet runs)
autozlevs=.false.
% pressure ceiling
pceil=40.0

###### Initial atmospheric temperature profile
#
# Type of initial temperature profile
#         ichoice=1   Constant Temperature:  T=tref
#         ichoice=2   Savidjari profile (as Seiff but with dT/dz=cte)
#         ichoice=3   Lindner (polar profile)
#         ichoice=4   inversion
#         ichoice=5   Seiff  (standard profile, based on Viking entry)
#         ichoice=6   constant T  +  gaussian perturbation (levels)
#         ichoice=7   constant T  + gaussian perturbation (km)
#         ichoice=8   Read in an ascii file "profile"
ichoice=5
# Reference temperature tref (K)
tref=200
# Add a perturbation to profile if isin=1
isin=0
# peak of gaussian perturbation (for ichoice=6 or 7)
pic=26.522
# width of the gaussian perturbation (for ichoice=6 or 7)
largeur=10
# height of the gaussian perturbation (for ichoice=6 or 7)
hauteur=30.

# some definitions for the physics, in file 'callphys.def'
INCLUDEDEF=callphys.def
\end{verbatim}
}
