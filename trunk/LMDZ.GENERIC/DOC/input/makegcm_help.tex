{\footnotesize
\begin{verbatim}
makegcm [Options] prog


The makegcm script:
-------------------

1. compiles a series of subroutines located in the $LMDGCM/libf
 sub-directories.
 The objects are then stored in the libraries in $LIBOGCM.

2. then, makegcm compiles program prog.f located by default in
$LMDGCM/libf/dyn3d and makes the link with the libraries.

Environment Variables '$LMDGCM' and '$LIBOGCM'
 must be set as environment variables or directly
 in the makegcm file.

The makegcm command is used to control the different versions of the model
 in parallel, compiled using the compilation options 
 and the various dimensions, without having to recompile the whole model.

The FORTRAN libraries are stored in directory $LIBOGCM.


OPTIONS:
--------

The following options can either be defined by default by editing the
makegcm "script", or in interactive mode:

-d imxjmxlm  where im, jm, and lm are the number of longitudes,
             latitudes and vertical layers respectively.

-t ntrac   Selects the number of tracers present in the model

             Options -d and -t overwrite file 
             $LMDGCM/libf/grid/dimensions.h
             which contains the 3 dimensions of the
             horizontal grid 
             im, jm, lm plus the number of tracers passively advected
             by the dynamics ntrac,
             in 4 PARAMETER FORTRAN format 
             with a new file:
             $LMDGCM/libf/grid/dimension/dimensions.im.jm.lm.tntrac
             If the file does not exist already
             it is created by the script
             $LMDGCM/libf/grid/dimension/makdim

-p PHYS    Selects the set of physical parameterizations
           you want to compile the model with.
           The model is then compiled using the physical
           parameterization sources in directory:
            $LMDGCM/libf/phyPHYS

-g grille  Selects the grid type.
           This option overwrites file
           $LMDGCM/libf/grid/fxyprim.h
           with file
           $LMDGCM/libf/grid/fxy_grille.h
           the grid can take the following values:
           1. reg - the regular grid
           2. sin - to obtain equidistant points in terms of sin(latitude)
           3. new - to zoom into a part of the globe

-O "compilation options" set of fortran compilation options to use

-include path
           Used if the subroutines contain #include files (ccp) that 
           are located in directories that are not referenced by default.

-adjnt     Compiles the adjoint model to the dynamical code.

-filtre  filter
           To select the longitudinal filter in the polar regions.
           "filter" corresponds to the name of a directory located in
           $LMDGCM/libf. The standard filter for the model is "filtrez"
           which can be used for a regular grid and for a  
           grid with longitudinal zoom.

-link "-Ldir1 -lfile1 -Ldir2 -lfile2 ..."
           Adds a link to FORTRAN libraries
           libfile1.a, libfile2.a ... 
           located in directories dir1, dir2 ...respectively
           If dirn is a directory with an automatic path 
           (/usr/lib ... for example) 
           there is no need to specify  -Ldirn.

\end{verbatim}
}
