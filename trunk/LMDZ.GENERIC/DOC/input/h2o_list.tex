{\footnotesize
\begin{verbatim}
## Orbit / general options
## ~~~~~~~~~~~~~~~~~~~~~~~
# Run with or without tracer transport ?
tracer    = .true.
# Diurnal cycle ?  if diurnal=false, diurnally averaged solar heating
diurnal   = .true.
# Seasonal cycle ? if season=false, Ls stays constant, to value set in "start"
season    = .true.
# Tidally resonant orbit ? must have diurnal=false, correct rotation rate in newstart
tlocked   = .false.
# Tidal resonance ratio ? ratio T_orbit to T_rotation
nres      = 10
# Write some more output on the screen ?
lwrite    = .false.
# Save statistics in file "stats.nc" ?
callstats = .true.
# Test energy conservation of model physics ?
enertest  = .true.

## Radiative transfer options
## ~~~~~~~~~~~~~~~~~~~~~~~~~~
# call radiative transfer?
callrad    = .true.
# the rad. transfer is computed every "iradia" physical timestep
iradia     = 4
# call multilayer correlated-k radiative transfer ?
corrk      = .true.
# folder in which correlated-k data is stored ?
corrkdir   = CO2_H2Ovar
# call visible gaseous absorption in radiative transfer ?
callgasvis = .true.
# Include Rayleigh scattering in the visible ?
rayleigh   = .true.
# Characteristic planetary equilibrium (black body) temperature
# This is used only in the aerosol radiative transfer setup. (see aerave.F)
tplanet    = 215.
# Output spectral OLR in 1D/3D?
specOLR    = .false.
# Output global radiative balance in file 'rad_bal.out' - slow for 1D!!
meanOLR    = .true.
# Variable gas species: Radiatively active ?
varactive  = .true.
# Variable gas species: Fixed vertical distribution ?
varfixed   = .false.
# Variable gas species: Saturation percentage value at ground ?
satval     = 0.0

## Star type
## ~~~~~~~~~
startype = 1
# ~~~~~~~~~~~~~~~~~~~~~~~~~~~~~~~~~~~~~~~~~~~~~~~~~~~~~~~~~~
# The choices are:
#
#	    startype = 1		Sol        (G2V-class main sequence)
#    	startype = 2		Ad Leo     (M-class, synthetic)
#       startype = 3        GJ644
#       startype = 4        HD128167
# ~~~~~~~~~~~~~~~~~~~~~~~~~~~~~~~~~~~~~~~~~~~~~~~~~~~~~~~~~~~
# Stellar flux at 1 AU. Examples:
# 1366.0 W m-2		Sol today
# 1024.5 W m-2		Sol today x 0.75 = weak early Sun
# 18.462 W m-2		The feeble Gl581
# 19.960 W m-2		Gl581 with e=0.38 orbital average
Fat1AU = 1024.5

## Tracer and aerosol options
## ~~~~~~~~~~~~~~~~~~~~~~~~~~
# Gravitational sedimentation of tracers (just H2O ice for now) ?
sedimentation = .false.

## Other physics options
## ~~~~~~~~~~~~~~~~~~~~~
# call turbulent vertical diffusion ?
calldifv = .true.
# call convective adjustment ?
calladj  = .true.
# call thermal conduction in the soil ?
callsoil = .true.

#########################################################################
## extra non-standard definitions for Early Mars
#########################################################################

## Tracer and aerosol options
## ~~~~~~~~~~~~~~~~~~~~~~~~~~
# Fixed aerosol distributions?
aerofixed     = .false.
# Varying H2O cloud fraction?
CLFvarying    = .false.
# H2O cloud fraction?
CLFfixval     = 1.0
# number mixing ratio of CO2 ice particles
Nmix_co2      = 100000.
# number mixing ratio of water ice particles
Nmix_h2o      = 100000.

## Water options
## ~~~~~~~~~~~~~
# Model water cycle
water         = .true.
# Model water cloud formation
watercond     = .true.
# Model water precipitation (including coagulation etc.)
waterrain     = .true.
# WATER: Precipitation threshold (simple scheme only) ?
rainthreshold = 0.0011
# Include hydrology ?
hydrology     = .true.
# H2O snow (and ice) albedo ?
albedosnow    = 0.5
# Maximum sea ice thickness ?
maxicethick   = 0.05
# Freezing point of seawater (degrees C) ?
Tsaldiff      = 0.0
# Evolve surface water sources ?
sourceevol    = .true.

## CO2 options
## ~~~~~~~~~~~
# gas is non-ideal CO2 ?
nonideal      = .false.
# call CO2 condensation ?
co2cond       = .true.
# Set initial temperature profile to 1 K above CO2 condensation everywhere?
nearco2cond   = .false.
\end{verbatim}
}
