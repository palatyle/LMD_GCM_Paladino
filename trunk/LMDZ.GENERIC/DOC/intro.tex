\chapter{Introduction}

\selectlanguage{english}
This document is a user manual for the Generic Climate Model
developed by the Laboratoire de M\'et\'eorologie
Dynamique of the CNRS in Paris.
It corresponds to the version of the model available since January 2011,
that includes the new dynamic code lmdz3.3
and input and output data in NetCDF format.
The physical part includes generalized correlated-k radiative transfer,
generalized tracer transport, and a water cycle that includes water vapour and ice transport,
radiative and thermodynamic effects, and simple hydrology.

Chapter~\ref{sc:apercu} of this document, to be read before any of the others,
describes the main features of the model.
The model is divided into two relatively independent parts:
(1) The hydrodynamic code, which integrates the fluid mechanical \emph{primitive equations} in time
over the globe, and (2) the physical parameterizations, which include the radiative transfer, tracer transport / evolution,
and surface-atmosphere interaction. It is followed by a list of references for anyone requiring a detailed
description of the physics and the numerical formulation of the parameterizations (Chapter~\ref{sc:phystd}).

For your {\bf first contact with the model}, Chapter~\ref{loc:contact1} guides the user through a practice simulation
(choosing the initial states and parameters and  visualizing the output files). The document then describes the code used for the model, including a user computer manual for compiling and running it (Chapter~\ref{sc:info}).

Chapter~\ref{sc:io} describes the input/output data of the model. The input files are the files needed to initialize the model (state of the atmosphere at instant $t0$ as well as a dataset of boundary conditions). The output files are ``historical files", archives of the atmospheric flow history as simulated by the model, the ``diagfi files", the ``stats files'', the daily averages, and so on. Common ways of editing or visualizing these files (editor ``ncdump" and the graphics software ``grads") are also explained.  Chapter~\ref{sc:water} explains how to run a simulation that includes the water cycle. Finally, Chapter~\ref{sc:rcm1d} will help you to use a 1-dimensional version of the model, which may be a simpler tool for some analysis work.
