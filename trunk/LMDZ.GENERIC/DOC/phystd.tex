\chapter{Physical parameterizations of the generic model: some references}

\label{sc:phystd}

\section{General}

The Generic Climate Model uses a large number of physical parameterizations based on various scientific theories. Some also use specific numerical methods. A list of these parameterizations is given below, along with the most appropriate references for each one. Most of these documents can be found at\\ 
\verb+http://www.lmd.jussieu.fr/mars.html+.

\paragraph{General references:}
No documents attempt to give a complete scientific description of the current
version of the GCM. Here's a reference to a Mars GCM description:
\begin{itemize}
\item  {\it Forget et al.} [1999] (article
published in the JGR)
\item  ``Updated Detailed Design Document for the Model''
(ESA contract, Work Package 6, 1999, available on the web)
which is simply a compilation of the preceding article with
a few additions that were published separately.
\end{itemize}

\nocite{Forg:99}

\section{Radiative transfer}

The radiative transfer parameterizations are used to calculate the heating
and cooling ratios in the atmosphere and the radiative flux at the surface.

[TO WRITE: IMPORTANT SECTION - REFERENCES HERE ARE FOR MARS ONLY]

%\end{itemize}


\subsection{\bf Absorption/emission and diffusion by dust:}

\subsubsection*{Dust spatial distribution}
 (\verb+ dustopacity+)

\begin{itemize}
\item Vertical distribution and description of ``MGS'' and ``Viking'' scenarios
in the ESA report {\it Mars Climate Database V3.0 Detailed Design Document}
by Lewis et al. (2001), available on the web.
\item For the ``MY24'' scenario, dust distribution obtained from assimilation
of TES data is used (and read via the \verb+readtesassim+ routine).
\end{itemize}
\nocite{Forg:99,Lewi:99}

\subsubsection*{Thermal IR radiation}
 (\verb+ lwmain+)
\begin{itemize}
\item Numerical method:  {\it Toon et al.} [1989]
\item Optical properties of dust:  {\it Forget} [1998]
\nocite{Toon:89,Forg:98grl}
\end{itemize}

\subsubsection*{Solar radiation}
 (\verb+ swmain+)
\begin{itemize}
\item Numerical method: {\it Fouquart and Bonel} [1980]
\nocite{Fouq:80}

\item Optical properties of dust:
see the discussion in {\it Forget et al. } [1999], which quotes
 {\it Ockert-Bell et al.} [1997] and {\it Clancy and Lee} [1991].
\nocite{Ocke:97,Clan:91}
\end{itemize}

\section{Subgrid atmospheric dynamical processes}

\subsection{Turbulent diffusion in the upper layer}
 (\verb+ vdifc+)

\begin{itemize}
\item Implicit numerical scheme in the vertical:
see the thesis of Laurent Li (LMD, Universit\'e Paris 7, 1990), Appendix C2.

\item Calculation of the turbulent diffusion coefficients:
{\it Forget et al. } [1999].
\end{itemize}

\subsection{Convection}
 (\verb+ convadj+)

See {\it Hourdin et al.} [1993]
\nocite{Hour:93}

\section{Surface thermal conduction}
 (\verb+soil+)

Thesis of Fr\'ed\'eric Hourdin (LMD, Universit\'e Paris 7, 1992) :
 section 3.3 (equations) and Appendix A (Numerical scheme).

\section{CO$_2$ Condensation}

In {\it Forget et al.} [1998] (article published in Icarus): \\
- Numerical method for calculating the condensation and sublimation levels
at the surface and in the atmosphere (\verb+ newcondens+)
 explained in the appendix.
\\
- Description of the numerical scheme for calculating the evolution of CO$_2$
snow emissivity (\verb+co2snow+) explained in section 4.1
\nocite{Forg:98}

\section{Tracer transport and sources}
\begin{itemize}
\item ``Van-Leer'' transport scheme used in the dynamical part
(\verb+ tracvl+ and  \verb+ vlsplt+ in the dynamical part):
{\it Hourdin and Armengaud} [1999] \nocite{Hour:99}

\item Transport by turbulent diffusion  (in \verb+ vdifc+), convection
(in  \verb+ convadj+), sedimentation  (\verb+ sedim+),
dust lifting by winds (\verb+ dustlift+) :
see note ``Preliminary design of dust lifting and transport in the Model''
(ESA contract, Work Package 4, 1998, available on the web).


%\item Simplified water cycle (source in {\tt vdifc}, {\tt
%watercloud}) : and also see the Maitrise study by Delphine Nobileau, LMD, 2000.
\item {\bf Watercycle}, see {\it Montmessin et al.} [2004]
\nocite{Mont:04jgr}

\end{itemize}


