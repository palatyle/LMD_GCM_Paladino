\chapter{Water Cycle Simulation}

\label{sc:water}

To simulate the water cycle with the LMD Generic Model:

\begin{itemize}
\item In {\tt callphys.def}, set tracer to true: {\tt tracer=.true.}. In the radiative
transfer sub-section, chose an appropriate correlated-k database that includes the effect
of water vapour (e.g. {\tt corrkdir=CO2H2Ovar}), and set {\tt varactive=.true.}, {\tt varfixed=.false.}.
In the water cycle sub-section you can chose various parameters - see below for a standard example.

\input{input/h2o_list.tex}

\item You need to compile with at least 2 tracers. If you don't have CO2 clouds,
dust or other tracers, compilation is done with the command lines:
\begin{verbatim}
makegcm -d 64x48x20 -t 2 -p std -b 32x36 newstart
\end{verbatim}
\begin{verbatim}
makegcm -d 64x48x20 -t 2 -p std -b 32x36 gcm
\end{verbatim}

Of course, you will also need an appropriate {\tt traceur.def} file
indicating you will use tracers {\tt h2o\_vap} and {\tt h2o\_ice};
if you only run with 2 tracers, then the contents of the {\tt traceur.def}
file should be:
\begin{verbatim}
2
h2o_ice
h2o_vap
\end{verbatim}
Note that the order in which tracers are set in the {\tt traceur.def} file
is not important.

\item {\bf Run} \\ \\
Same as usual. Just make sure that your start files contains
the initial states for water, with an initial state for water vapour / ice
in the atmosphere and ice / liquid on the surface.

\end {itemize}
