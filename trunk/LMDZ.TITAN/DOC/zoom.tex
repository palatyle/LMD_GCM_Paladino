\chapter{Zoomed simulations}

\label{sc:zoom}

The LMD GCM can use a zoom to enhance the resolution locally.
In practice, one can
increase the latitudinal resolution on the one hand,
and the longitudinal resolution on
the other hand. 

\section{To define the zoomed area}

The zoom is defined in {\tt run.def}.
Here are the variables that you want to set:

\begin{itemize}
\item East longitude (in degrees) of zoom center {\tt clon}
\item latitude (in degrees) of zoom center {\tt clat}
\item zooming factors, along longitude {\tt grossismx}.
      {\it Typically 1.5, 2 or even 3 (see below)}
\item zooming factors, along latitude {\tt grossismy}. {\it Typically 1.5, 2
or even 3 (see below)}
\item {\tt fxyhypb}:
      {\it {\bf must be set to "T" for a zoom}, whereas it must be F otherwise}
\item extention in longitude  of zoomed area {\tt dzoomx}.
      This is the total
      longitudinal extension of the zoomed region (degree). \newline
      {\it It is recommended that {\tt grossismx} $\times$
      {\tt dzoomx} $< 200^o$}
\item extention in latitude of the zoomed region {\tt dzoomy}.
      This is the total
      latitudinal extension of the zoomed region (degree). \newline
      {\it It is recommended that {\tt
      grossismy} $\times$ {\tt dzoomy} $< 100^o$}
\item stiffness of the zoom along longitudes {\tt taux}.
      2 is for a smooth transition in
      longitude, more means sharper transition.
\item stiffness of the zoom along latitudes {\tt taux}.
      2 is for a smooth transition in
      latitude, more means sharper transition.
\end{itemize}

\section{Making a zoomed initial state}

One must start from an initial state archive {\tt start\_archive.nc}
obtained from a previous
simulation (see section~\ref{sc:newstart})
Then compile and run {\tt newstart.e} {\bf using the {\tt run.def}
file designed for the zoom}.

After running {\tt newstart.e}. The zoomed grid may be visualized
using grads, for instance.
Here is a grads script that can be used to map the grid above a topography
map:

\begin{verbatim}
set mpdraw off
set grid off
sdfopen restart.nc
set gxout grid
set digsiz 0
set lon -180 180
d ps
close 1
*** replace the path to surface.nc in the following line:
sdfopen  /u/forget/WWW/datagcm/datafile/surface.nc
set lon -180 180
set gxout contour
set clab off
set cint 3
d zMOL
\end{verbatim}


\section{Running a zoomed simulation and stability issue}

\begin{itemize}

\item {\bf dynamical timestep}
Because of their higher resolution, zoomed simulation requires a higher
timestep.
Therefore in {\tt run.def}, the number of dynamical timestep per day
{\tt day\_step} must be increased by more than {\tt grossismx} or
{\tt grossismy} (twice that if necessary).
However, you can keep the same physical timestep (48/sol) and thus increase
 {\tt iphysiq} accordingly ({\tt iphysiq = day\_step/48}).

\item It has been found that when zooming in longitude, on must set
{\tt ngroup=1} in
{\tt dyn3d/groupeun.F}. Otherwise the run is less stable.

\item The very first initial state made with {\tt newstart.e} can be noisy and
dynamically unstable.
It may be necessary to strongly increase the intensity of the
dissipation and increase {\tt day\_step} in {\tt run.def} for 1 to 3 sols,
and then use less strict values. 

\item If the run remains very unstable and requires too much dissipation
or a too small timestep, a good tip to help stabilize the model
is to decrease the vertical extension of your run and the number of
layer (one generally zoom to study near-surface process, so 20 to 22
layers and a vertical extension up to 60 or 80 km is usually enough).

\end{itemize}













 



