{\footnotesize
\begin{verbatim}

#------------------------------
# Parametres de controle du run
#------------------------------

# Nombre de jours d'integration
     nday=669

# nombre de pas par jour (multiple de iperiod) ( ici pour  dt = 1 min )
 day_step = 960

# periode pour le pas Matsuno (en pas)
  iperiod=5

# periode de sortie des variables de controle (en pas)
  iconser=120

# periode d'ecriture du fichier histoire (en jour)
    iecri=100

# periode de stockage fichier histmoy (en jour)
 periodav=60.

# periode de la dissipation (en pas)
  idissip=5

# choix de l'operateur de dissipation (star ou  non star )
 lstardis=.true.

# avec ou sans coordonnee hybrides
 hybrid=.true.

# nombre d'iterations de l'operateur de dissipation   gradiv
nitergdiv=1

# nombre d'iterations de l'operateur de dissipation  nxgradrot
nitergrot=2

# nombre d'iterations de l'operateur de dissipation  divgrad
   niterh=2

# temps de dissipation des plus petites long.d ondes pour u,v (gradiv)
 tetagdiv=10000.

# temps de dissipation des plus petites long.d ondes pour u,v(nxgradrot)
 tetagrot=10000.

# temps de dissipation des plus petites long.d ondes pour  h ( divgrad)
 tetatemp=10000.

# coefficient pour gamdissip
  coefdis=0.

# choix du shema d'integration temporelle (Matsuno ou Matsuno-leapfrog)
  purmats=.false.

# avec ou sans physique
   physic=.true.

# periode de la physique (en pas)
  iphysiq=20

# choix d'une grille reguliere
  grireg=.true.

# frequence (en pas) de l'ecriture du fichier diagfi
 ecritphy=1920

# longitude en degres du centre du zoom
   clon=63.

# latitude en degres du centre du zoom
   clat=0.

# facteur de grossissement du zoom,selon longitude
  grossismx=1.

# facteur de grossissement du zoom ,selon latitude
 grossismy=1.

#  Fonction  f(y)  hyperbolique  si = .true.  , sinon  sinusoidale
  fxyhypb=.false.

# extension en longitude  de la zone du zoom  ( fraction de la zone totale)
   dzoomx= 0.

# extension en latitude de la zone  du zoom  ( fraction de la zone totale)
   dzoomy=0.

#  raideur du zoom en  X
    taux=2.

#  raideur du zoom en  Y
    tauy=2.

#  Fonction  f(y) avec y = Sin(latit.) si = .TRUE. ,  Sinon  y = latit.
  ysinus= .false.

# Avec sponge layer
  callsponge  = .true.

# Sponge:  mode0(u=v=0), mode1(u=umoy,v=0), mode2(u=umoy,v=vmoy)
  mode_sponge= 2

# Sponge:  hauteur de sponge (km)
  hsponge= 90

# Sponge:  tetasponge (secondes)
  tetasponge = 50000

# some definitions for the physics, in file 'callphys.def'
INCLUDEDEF=callphys.def


\end{verbatim}
}
