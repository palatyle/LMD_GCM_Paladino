%
%
%     Test comportement LATEX et LATEX2HTML
%
%
%
\documentclass[dvips]{report}
\usepackage{makeidx}
\usepackage{a4}
\usepackage{graphicx}
%
% Pour definir un paragraph avec retour a la ligne
\usepackage{loc}
%
\usepackage{html,htmllist}
% \usepackage{times}
\scrollmode
% use %sort -f -u manual.idx > manual.index for a primitive index
%
%  NOTE:  You must use LaTeX2e in order to process this document
%	If you do not have LaTeX2e, a PostScript version
%	(manual.ps) is included with this distribution.
%
%%%%%%%%%%%%%%%%%%% No changes beyond this point %%%%%%%%%%%%%%%%%%%%%%%%%%%%%

\makeindex
\sloppy
% Transforms the \indexentry generated by \makeindex in the concepts.idx
% file into a form understood by the theindex environment.
% See below on how to produce an index.
%
\newcommand{\latextohtml}{\LaTeX 2\texttt{HTML}}
\newcommand{\fn}[1]{{\ttfamily #1}}	% file names
\newcommand{\Email}[1]{{\ttfamily <#1>}}% file names
\newcommand{\HTML}[1]{{\ttfamily <#1>}}%  HTML tag
\newcommand{\Meta}[1]{\texttt{<\emph{#1}>}}%  Meta string
\newcommand{\indexentry}[2]{\item #1 #2}
\def\contrat{contrat numero N de l'ESA}
\newcommand{\onlinedoc}{http://www-dsed.llnl.gov/files/programs/unix/latex2html/manual/}
\newcommand{\patches}{http://www-dsed.llnl.gov/files/programs/unix/latex2html}
\newcommand{\sourceA}{ftp://www-dsed.llnl.gov/files/programs/unix/latex2html/sources/latex2html-96.1.tar.gz}
\newcommand{\sourceB}{ftp://ftp.mpn.com/pub/nikos/latex2html-96.1.tar.gz}
\newcommand{\sourceC}{ftp://ftp.rzg.mpg.de/outgoing/latex2html-96.1.tar.gz}
\newcommand{\CTAN}{tex-archive/support/latex2html}

%\setlength{\textwidth}{5.5in}
%\setlength{\textwidth}{16cm}
%\setlength{\changebarwidth}{1pt}
%\addtolength{\oddsidemargin}{-1in}
%\addtolength{\oddsidemargin}{-2cm}
%\addtolength{\evensidemargin}{-1in}
%\addtolength{\evensidemargin}{-2cm}
 
% numero de la derniere section numerotee
\setcounter{secnumdepth}{3}
% numero de la derniere section a apparaitre dans la table des matieres
\setcounter{tocdepth}{3}
 

\begin{document}
\bibliographystyle{abbrv}

\newcommand{\der}[2]{\frac{\partial #1 }{\partial #2} }

%termes lations
\def\ie{{\em i.~e.}}
\def\etal{{\em et~al.}}
\def\apriori{{\em a~priori}}
\def\apost{{\em a~posteriori}}
\def\afort{{\em a~fortiori}}
\def\etc{{\em etc.}}
\def\insitu{{\em in--situ}}
\def\adoc{{\em ad~hoc}}

\def\sc#1{Section~\ref{sc:#1}}
\def\an#1{Annexe~\ref{an:#1}}
\def\ch#1{Chapitre~\ref{ch:#1}}
\def\fg#1{Fig.~\ref{fg:#1}}
\def\fgs#1{Figs.~\ref{fg:#1}}
\def\eq#1{Eq.~\ref{eq:#1}}
\def\eqs#1{Eqs.~\ref{eq:#1}}
\def\tb#1{Table~\ref{tb:#1}}
\newcommand{\av}[2]{{\overline{#1}}^{ #2 }}
\newcommand{\avg}[1]{\left< #1 \right>}
\def\cd{C_D}
% definitions de Macros pour tout l'article
\def\co{$CO_2$}
\def\eq#1{Eq.~\ref{eq:#1}}
\def\fig#1{Fig.~\ref{fg:#1}}
\newcommand{\dep}[1]{\left( #1 \right) }
\newcommand{\depb}[1]{\left[ #1 \right] }
\newcommand{\depc}[1]{\left\{ #1 \right\} }
\def\coband{$CO_2\ 15\mu$m~band}
\def\ps{p_S}
\def\etal{{\em~et al. }}
\def\insitu{{\em in--situ }}
\def\t{\theta}
\def\dnw{^{\downarrow}}


%%%%%%%%%%%%%%%%%%%%%%%%%%%%%%%%%%%%%%%%%%%%%%%%%%%%%%%%%%%%%%%%%%%%%%%
%%%%%%%%%%%%%%%%%%%%%%%%%%%%%%%%%%%%%%%%%%%%%%%%%%%%%%%%%%%%%%%%%%%%%%%
%    variables definies pour doc Terre ou Mars

\def\planete{mars}
\def\italique{{\it mars}}
\def\planeteX{mars/}
\def\STARTFI{startfi}
%%%%%%%%%%%%%%%%%%%%%%%%%%%%%%%%%%%%%%%%%%%%%%%%%%%%%%%%%%%%%%%%%%%%%%%
%%%%%%%%%%%%%%%%%%%%%%%%%%%%%%%%%%%%%%%%%%%%%%%%%%%%%%%%%%%%%%%%%%%%%%%




planete sans espace -\planete-\\
italique -\italique-\\
planete en italique -{\it \planete}-\\

planeteX sans espace -\planeteX-\\
planete avec espace -\planete -\\
planete avec /  -\planete/coucou\\
STARTFI sans espace -\STARTFI-\\
STARTFI avec espace -\STARTFI -\\


% lorsqu'on appelle une variable definie au prealable dans une nouvelle
%commande, latex2html ajoute un blanc apres cette variable.

\def\OK#1#2{\planete/#1/#2.html}
ecrit OK: \OK{dyn3d}{lectba}\\
ecrit OK: mars/dyn3d/lectba.html\\

ATTENTION: latex2html n'aime pas les \def\nom ou nom contient un
chiffre!\\


\def\sourceF#1#2{\htmladdnormallink{#2}{#1}}


\sourceF{dyn3d}{lectba}

\end{document}
