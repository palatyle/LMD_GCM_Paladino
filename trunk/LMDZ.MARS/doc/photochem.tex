\chapter{Photochemical Module}

\label{sc:photochem}

The LMD GCM includes a photochemical module, which allows to compute 
the atmospheric composition.

\begin{itemize}

\item 14 chemical species are included: CO$_2$ (background gas), CO, O, 
O({\em $^1D$}), O$_2$, O$_3$, H, H$_2$, OH, HO$_2$, H$_2$O$_2$, N$_2$,
Ar (inert) and H$_2$O.

\item In {\tt callphys.def}, set tracer to true {\tt tracer=.true.}.
Use the same options as shown below for the tracer part of {\tt callphys.def}.
You need to set {\tt photochem=.true.}, 
and to include the water cycle ({\tt water=.true.},
{\tt sedimentation=.true.}; see Chapter~\ref{sc:water}), because 
composition is extremely dependent on the water vapor abundance.
%{\footnotesize

\begin{verbatim}                                                                               
## Tracer (dust water, ice and/or chemical species) options (used if tracer=T):
## ~~~~~~~~~~~~~~~~~~~~~~~~~~~~~~~~~~~~~~~~~~~~~~~~~~~~~~~~~~~~~~~~~~~~~~~~~~~
# DUST: Transported dust ? (if >0, use 'dustbin' dust bins)
dustbin    = 0
# DUST: Radiatively active dust ? (matters if dustbin>0)
active  = .false.
# DUST: use mass and number mixing ratios to predict dust size ?
# (must also have dustbin=1)
doubleq   = .false.
# DUST: lifted by GCM surface winds ?
lifting = .false.
# DUST: lifted by dust devils ?
callddevil = .false.
# DUST: Scavenging by CO2 snowfall ?
scavenging = .false.
# DUST/WATERICE: Gravitationnal sedimentation ?
sedimentation = .true.
# WATERICE: Radiatively active transported atmospheric water ice ?
activice   = .false.
# WATER: Compute water cycle
water = .true.
# WATER: current permanent caps at both poles. True IS RECOMMENDED
#        (with .true., North cap is a source of water and South pole
#         is a cold trap)
caps  = .true.
# PHOTOCHEMISTRY: include chemical species 
photochem  = .true.
\end{verbatim}
}


\item You will need the up-to-date file {\tt jmars.yyyymmdd} 
(e.g. {\tt jmars.20140930}), which contains the photodissociation rates.
It should be in the {\it datafile} directory in which are stored
datafiles used by the GCM (the path to these files is set in file
{\tt callphys.def}).

\item {\bf Settings} \\ \\
Compile as usual, what really matters are the options in {\tt callphys.def}
and {\tt traceur.def}.
You need at least 19 tracers:
13 chemical species (co2, co, o, o(1d), o2, o3, h, h2, oh, ho2, h2o2, n2, ar)
along with water vapor (h2o\_vap), water ice (h2o\_ice), related condensation
nuclei moments (ccn\_mass and ccn\_number), and dust moments
(dust\_mass and dust\_number).\\
the {\tt traceur.def} file should thus be something like:
\begin{verbatim}
19
co2
co
o
o1d
o2
o3
h
h2
oh
ho2
h2o2
n2
ar
dust_mass
dust_number
h2o_vap
h2o_ice
ccn_mass
ccn_number
\end{verbatim}


\item {\bf Run} \\ \\
Same as usual. Just make sure that your start files contains the correct number
of tracers. If you need to initialize the composition, you can run
{\bf newstart} and use the options 
\begin{description}
\item - ini\_q: the 15 tracers are initialized, including water ice and vapor.
\item - ini\_q-h2o: the 13 chemical species are initialized, water ice is put 
to zero, and water vapor is kept untouched.
\item - ini\_q-iceh2o: the 13 chemical species are initialized, water ice and 
vapor are kept untouched.
\end{description}
The initialization is done with the files {\tt atmosfera\_LMD\_may.dat} and 
{\tt atmosfera\_LMD\_min.dat}, which should also be found in the 
{\it datafile} directory.

\item {\bf Outputs} \\ \\
The outputs can be done from the {\tt aeronomars/calchim.F} routine for the 14
chemical species. The variables put in the {\tt diagfi.nc} and 
{\tt stats.nc} files are
labeled (where {\it name} is the name of the chemical species, e.g. co2):
\begin{description}
\item - n\_{\it name}: local density (in molecule cm$^{-3}$, 
3-dimensional field)
\item - c\_{\it name}: integrated column density (in molecule cm$^{-2}$, 
2-dimensional field) 
\end{description}

\end{itemize}


