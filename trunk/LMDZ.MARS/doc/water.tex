\chapter{Water Cycle Simulation}

\label{sc:water}

In order to simulate the water cycle with the LMD GCM:

\begin{itemize}
\item In {\tt callphys.def}, set tracer to true: {\tt tracer=.true.}.

\item It is best to run with a semi-interactive dust (in order to
better represent the evolution available condensation nuclei) and thus set
{\tt dustbin = 2}, and use corresponding additional tracers ({\tt dust\_mass}
and {\tt dust\_number} which correspond to the first two moments of the dust
distribution).

\item Use the same options as given in the sample {\tt callphys.def.watercycle}
file provided in {\tt LMDZ.MARS/deftank}.

\item {\bf Settings}
You need to run with at least 4 tracers (if you don't have dust
({\tt dustbin=0}) or other chemical species ({\tt photochem=F}),
but 7 is recommended, and an appropriate {\tt traceur.def} for running with
combined CO22, dust and water cycle should be:
\begin{verbatim}
7
co2
dust_mass
dust_number
h2o_vap
h2o_ice
ccn_mass
ccn_number
\end{verbatim}
Note that the order in which tracers are set in the {\tt tracer.def} file
is not important.

\item {\bf Run} \\ \\
Same as usual. Just make sure that your start files contains
the initial states for all the tracers you use (or else compile
and run {\bf newstart.e} to initialize them).

\end {itemize}









 



