\chapter{1D version of the Mars model}

\label{sc:testphys1d}

The physical part of the model can be used to generate realistic 
1-D simulations (one atmosphere column). 
In practice, the simulation is controlled from a main program called
\verb+ testphys1d.F+ which, after initialization, then calls the master
subroutine of the physics \verb+ physiq.F+ described in the preceeding
chapters.

\section{Compilation} 
To compile the Martian model in 1-D with
29 layers, simply type
(in compliance with the makelmdz\_fcm function manual described in section
\ref{sc:compil1})

\begin{verbatim}
makelmdz_fcm -arch local -d 29 -p mars testphys1d
\end{verbatim}

You will find executable {\bf testphys1d\_29\_phymars\_seq.e}
(the compiled model)
in the {\tt bin} subdirectory from which you ran the makelmdz\_fcm command.

\section{1-D runs and input files}

The 1-D model does not use an initial state file (the simulation must be long
enough to obtain a balanced state). Thus, to generate a simulation simply
type:

\begin{verbatim}
./testphys1d.e
\end{verbatim}

The following example files are available in the {\tt deftank} directory
(copy them into your working directory first):

- {\bf callphys.def}~: controls the options in the physics,
  just like for the 3D GCM.

- {\bf z2sig.def}~:
 controls the vertical discretization
 (no change needed, in general), functions as with the 3D GCM. 

- {\bf traceur.def}~:
 controls the tracer names (this file may not be present, as long
 as you run without tracers (option {\tt tracer=.false.} in
 callphys.def)

- {\bf run.def}~: controls the 1-D run parameters and initializations
(this is actally file {\tt run.def.1d} the {\tt deftank} directory,
which must be renamed {\tt run.def} to be read by the program).\\

The last file is different from the 3D GCM's {\tt run.def} input file,
as it contains options specific to the 1-D model, as shown in the example
below:
{\footnotesize
{\footnotesize
\begin{verbatim}
#-----------------------------------------------------------------------
# Run parameters for the rcm1d.e model
#-----------------------------------------------------------------------

#### Time integration parameters
#
# Initial date (in martian sols ; =0 at Ls=0)
day0=0
# Initial local time (in hours, between 0 and 24)
time=0
# Number of time steps per sol
day_step=48
# Number of sols to run
ndt =400

#### Physical parameters
#
# Surface pressure (Pa)
psurf=7000.
# Gravity (ms^-2)
g=3.72
# Molar mass of atmosphere (g)
mugaz=43.49
# Specific heat capacity of atmosphere?
cpp=744.5
# latitude (in degrees)
latitude=0.0
# orbital distance at perihelion (AU)
periastr=1.558
# orbital distance at aphelion (AU)
apoastr=1.558
# obliquity (degrees)
obliquit=0.0
# Solar zenith angle (degrees)
szangle=60.0

# Albedo of bare ground
albedo=0.2
# Emissivity of bare ground
emis=1.0
# Soil thermal inertia (SI)
inertia=400
# zonal eastward component of the geostrophic wind (m/s)
u=10.
# meridional northward component of the geostrophic wind (m/s)
v=0.
# Initial CO2 ice on the surface (kg.m-2)
co2ice=0
# hybrid vertical coordinate ? (.true. for hybrid and .false. for sigma levels)
hybrid=.false.
# autocompute vertical discretisation? (useful for exoplanet runs)
autozlevs=.false.
% pressure ceiling
pceil=40.0

###### Initial atmospheric temperature profile
#
# Type of initial temperature profile
#         ichoice=1   Constant Temperature:  T=tref
#         ichoice=2   Savidjari profile (as Seiff but with dT/dz=cte)
#         ichoice=3   Lindner (polar profile)
#         ichoice=4   inversion
#         ichoice=5   Seiff  (standard profile, based on Viking entry)
#         ichoice=6   constant T  +  gaussian perturbation (levels)
#         ichoice=7   constant T  + gaussian perturbation (km)
#         ichoice=8   Read in an ascii file "profile"
ichoice=5
# Reference temperature tref (K)
tref=200
# Add a perturbation to profile if isin=1
isin=0
# peak of gaussian perturbation (for ichoice=6 or 7)
pic=26.522
# width of the gaussian perturbation (for ichoice=6 or 7)
largeur=10
# height of the gaussian perturbation (for ichoice=6 or 7)
hauteur=30.

# some definitions for the physics, in file 'callphys.def'
INCLUDEDEF=callphys.def
\end{verbatim}
}

}
Note that, just as for the 3-D GCM {\tt run.def} file, input
parameters may be given in any order, or even not given at all
(in which case default values are used by the program).

Initial conditions for tracers can be provided as ASCII text files; when
{\bf testphys1d.e} runs, it looks for a {\tt profile\_*} file, where {\tt *}
stands for a tracer name (as given in the traceur.def file) to initialize
the tracer profile. The data in the file should be given as one value per
line, first line for the surface value (in kg/m2), followed by values for each
atmospheric layer (going from bottom to top of the atmosphere).
If no {\tt profile\_*} file is found, then the tracer values are initialized
to zero (except for the 'co2' tracer, which will be set to 0.95).

\section{Output data}

During the entire 1D simulation, you can obtain output data for any
variable from any physical subroutine by using subroutine \verb+ writediagfi+
(as when runing 3D simulations, but specifying the correct dimensions for the field) \\

Example of a call to subroutine {\tt writediagfi} requesting
temperature output:
(\verb+ ngrid+ horizontal point, \verb+ nlayer + layers, variable
\verb+ zt + called ``temp'' in K units):

\begin{verbatim}
      CALL WRITEDIAGFI(ngrid,'temp','Temperature','K',1,zt)
\end{verbatim}


