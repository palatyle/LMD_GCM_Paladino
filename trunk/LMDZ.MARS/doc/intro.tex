\chapter{Introduction}

\selectlanguage{english}
This document is a user manual for the General Circulation Model
of the Martian atmosphere developed by the Laboratoire de M\'et\'eorologie
Dynamique of the CNRS in Paris in collaboration with the Atmospheric and
Oceanic Planetary Physics sub-department in Oxford.
It corresponds to the version of the model available since November 2002,
that includes the new dynamic code lmdz3.3 
and the input and output data in NetCDF format.
The physical part has been available since June 2001,
including the NLTE radiative transfer code valid at up to 120~km,
tracer transport, the water cycle with water vapour and ice,
the "double mode" dust transport model, and with optional photochemistry
and extension in the thermosphere up to 250km. 

A more general, scientific description of the model without
tracers can be found in {\it Forget et al.} [1999].\\


Chapter~\ref{sc:apercu} of this document, to be read before any of the others,
describes the main features of the model.
The model is divided into two relatively independent parts:
(1) The hydrodynamic code, that is shared by all atmospheres (Earth,
Mars, etc.) that integrates the fluid mechanics equations in time
and on the globe, and (2) a set of Martian physical parameterizations,
including, for example, the radiative transfer calculation in the atmosphere
and the turbulence mix in the upper layer.

It is followed by a list of references for anyone requiring a detailed 
description of the physics and the numerical formulation of the 
parameterizations of the Martian physical part (Chapter~\ref{sc:phymars}).

For your {\bf first contact with the model}, chapter
\ref{loc:contact1} guides the user through a practice simulation
(choosing the initial states and parameters and  visualizing the output files).

The document then describes the programming code for the model,
including a user computer manual for compiling and running the model
(Chapter~\ref{sc:info}).

Chapter~\ref{sc:io} describes the input/output data of the model.
The input files are the files needed to initialize the model
(state of the atmosphere at the initial time $t0$ as well as a dataset of
boundary conditions)
and the output files are "time series",i.e.
records of the atmospheric flow evolution as simulated by the model,
the ``diagfi files", the ``stats files'',
the daily averages etc.
Some means to edit or visualize these files
(editor ``ncdump" and the graphics software ``grads") are also described. \\ 


Chapter~\ref{sc:water} explains how to run a simulation including the
water cycle.
Chapter~\ref{sc:photochem} illustrates how to run the model
with the photochemical module.
%Chapter~\ref{sc:photochem} and chapter~\ref{sc:thermosphere} detail
%respectively the optional photochemical module and the extension model
%to the thermosphere.

Finally, chapter~\ref{sc:testphys1d} will help you to use a 
1-dimensional version of the model,
which may be a simplier tool for some analysis work.





