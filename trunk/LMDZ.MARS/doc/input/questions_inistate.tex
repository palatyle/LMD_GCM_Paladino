{\footnotesize
\begin{verbatim}
 First set of questions:
 Modifications of variables in tab_cntrl:
 ~~~~~~~~~~~~~~~~~~~~~~~~~~~~~~~~~~~~~~
 (3)          day_ini : Initial day (=0 at Ls=0)
 (19)              z0 : default surface roughness (m)
 (21)       emin_turb :  minimal energy (PBL)
 (20)         lmixmin : mixing length (PBL)
 (26)         emissiv : ground emissivity
 (24 et 25)   emisice : CO2 ice max emissivity 
 (22 et 23)  albedice : CO2 ice cap albedos
 (31 et 32) iceradius : mean scat radius of CO2 snow
 (33 et 34) dtemisice : time scale for snow metamorphism
 (27)        tauvis : mean dust vis. reference opacity
 (35)         volcapa : soil volumetric heat capacity
 (18)        obliquit : planet obliquity (deg)
 (17)      peri_day : perihelion date (sol since Ls=0)
 (  )      peri_ls : perihelion date (Ls since Ls=0)
 (15)      periheli : min. sun-mars dist (Mkm)
 (16)      aphelie  : max. sun-mars dist (Mkm)

 Second set of questions :
 List of possible changes :
 ~~~~~~~~~~~~~~~~~~~~~~~~~~
 flat         : no topography ("aquaplanet")
 bilball      : uniform albedo and thermal inertia
 z0           : set a uniform surface roughness length
 coldspole    : cold subsurface and high albedo at      S.Pole
 qname        : change tracer name
 q=0          : ALL tracer =zero
 q=x          : give a specific uniform value to one    tracer
 q=profile    : specify a profile for a tracer
 freedust     : rescale dust to a true value
 ini_q        : tracers initialization for chemistry    and water vapour
 ini_q-h2o    : tracers initialization for chemistry    only
 composition  : change atm main composition: CO2,N2,Ar, O2,CO
 ini_h2osurf  : reinitialize surface water ice 
 noglacier    : Remove tropical H2O ice if |lat|<45
 watercapn    : H20 ice on permanent N polar cap 
 watercaps    : H20 ice on permanent S polar cap 
 wetstart     : start with a wet atmosphere
 isotherm     : Isothermal Temperatures, wind set to    zero
 co2ice=0     : remove CO2 polar cap
 ptot         : change total pressure
 therm_ini_s  : set soil thermal inertia to reference   surface values
 subsoilice_n : put deep underground ice layer in       northern hemisphere
 subsoilice_s : put deep underground ice layer in       southern hemisphere
 mons_ice     : put underground ice layer according     to MONS derived data

\end{verbatim}
}
