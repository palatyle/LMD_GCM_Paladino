\chapter{Thermosphere}

\label{sc:thermosphere}

Some advice and comments about running the GCM with the thermosphere (roughly speaking, above 125~km):
\begin{itemize}
\item When running with the thermosphere, one must ensure that the model extends high enough. In practice runing with 49 levels (as defined in the default {\tt z2sig.def} file is necessary.
\item Example of definition files appropriate for simulating the thermosphere are provided in the
{\tt LMDZ.MARS/deftank} directory: {\tt callphys.def.MCD5}, {\tt run.def.64x48x49.MCD5} and {\tt traceur.def.MCD5} 
\item Chemistry in the thermosphere requires that file {\tt chemthermos\_reactionrates.def} (available in the
{\tt deftank} directory) be present in the directory where the GCM is run.
\item The main forcing in the thermosphere is the Extreme UV (EUV) input received from the Sun. Two settings are possible, depending on the value of flag {\tt solvarmod} (in {\tt callphys.def}):
 \begin{itemize}
 \item If {\tt solvarmod=0}, then the solar forcing is fixed throughout the simulation, using the E10.7 value set by the {\tt fixed\_euv\_value} flag.
 \item if {\tt solvarmod=1}, the the solar forcing evolves daily, following observed values of E10.7. For this case, one must also set the Mars Year (MY 23 to 33 are currently available) using the {\tt solvaryear} flag.
\end{itemize}
\end{itemize} 
