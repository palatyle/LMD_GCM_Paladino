\chapter{The physical parameterizations of the Martian model: some references}

\label{sc:phymars}

\section{General}

The Martian General Circulation Model uses a large number of physical
parameterizations based on various scientific theories
and some generated using specific numerical methods.

A list of these parameterizations is given below, along with the most
appropriate
references for each one. Most of these documents can be consulted at:
\verb+http://www-mars.lmd.jussieu.fr/mars/publi.html+.


\paragraph{General references:}
A document attempts to give a complete scientific description of the current
version of the GCM (a version without tracers): 
\begin{itemize}
\item  {\it Forget et al.} [1999] (article
published in JGR) 
\end{itemize}

\nocite{Forg:99}

\section{Radiative transfer}

The radiative transfer parameterizations are used to calculate the heating
and cooling ratios in the atmosphere and the radiative flux at the surface.

\subsection{\bf CO$_2$ gas absorption/emission:}
\subsubsection*{Thermal IR radiation} (\verb+ lwmain+)
\begin{itemize}
\item New numerical method, solution for the radiative transfer equation:
{\it Dufresne et al.} [2005]. 
\item Model validation and inclusion of the ``Doppler'' effect
(but using an old numerical formulation):
{\it Hourdin} [1992] (article).
\nocite{Hour:92,Hour:00b,Dufr:05}

\item At high altitudes, parameterization of the thermal radiative transfer
({\tt nltecool}) when the local thermodynamic balance is no longer valid
(e.g. within 0.1 Pa) : Lopez-Valverde et al. [2001] :
Report for the ESA available on the web
as: ``CO2 non-LTE cooling rate at
15-um and its parameterization for the Mars atmosphere''.

\end{itemize}

\subsubsection*{Absorption of near-infrared radiation}
 (\verb+ nirco2abs+)
\begin{itemize}
\item {\it Forget et al.} [1999]
\end{itemize}

\subsection{\bf Absorption/emission and diffusion by dust:}

\subsubsection*{Dust spatial distribution}
 (\verb+ aeropacity+)

\begin{itemize}
\item The method for semi-interactive dust vertical distribution
is detailed in {\it Madeleine et al.} [2011]
\item Vertical distribution and description of ``MGS'' and ``Viking'' scenarios
in the ESA report {\it Mars Climate Database V3.0 Detailed Design Document}
by Lewis et al. (2001), available on the web.
\item For the ``MY24''-``MY26'' scenarios, the dust distributions were
derived from observations made by
TES data is used. See technical note WP12.2.1 of ESA contract
Ref~ESA 11369/95/NL/JG(SC) "New dust scenarios for the Mars Climate Model : Martian Years
24-29", available online at
\verb+http://www-mars.lmd.jussieu.fr/WP2011/wp12.1.1.pdf+

\end{itemize}
\nocite{Lewi:99,Made:11}

\subsubsection*{Thermal IR radiation}
 (\verb+ lwmain+)
\begin{itemize}
\item Numerical method:  {\it Toon et al.} [1989] 
\item Optical properties of dust:  {\it Madeleine et al.} [2011] 
\nocite{Toon:89,Made:11}
\end{itemize}

\subsubsection*{Solar radiation}
 (\verb+ swmain+)
\begin{itemize}
\item Numerical method: {\it Toon et al.} [1989]
\nocite{Toon:89}

\item Optical properties of dust: 
see the discussion in {\it Madeleine et al.} [2011], which quotes
properties from {\it Wolff et al.} [2009].
\nocite{Made:11,Wolf:09}
\end{itemize}

\section{Subgrid atmospheric dynamical processes}

\subsection{Turbulent diffusion in the upper layer}
 (\verb+ vdifc+)

\begin{itemize}
\item Implicit numerical scheme in the vertical:
see the thesis of Laurent Li (LMD, Universit\'e Paris 7, 1990), Appendix C2.

\item Calculation of the turbulent diffusion coefficients:
{\it Forget et al. } [1999].

\item fluxes in the near-surface layer: {\it Colaitis et al.} [2012],
technical note WP13.1.3d of ESA contract
Ref~ESA 11369/95/NL/JG(SC) "New Mars Climate Model:
d) New convection and boundary layer schemes and their impact on
Mars meteorology", available online at 
\verb+http://www-mars.lmd.jussieu.fr/WP2011/wp13.1.3d.pdf+

\end{itemize}

\subsection{Convection}
 (\verb+ convadj+)
\begin{itemize}
\item For some details on the convective adjustement,
see {\it Hourdin et al.} [1993]
\item The thermals' mass flux scheme is described in
{\it Colaitis et al.} [2012],
technical note WP13.1.3d of ESA contract
Ref~ESA 11369/95/NL/JG(SC) "New Mars Climate Model:
d) New convection and boundary layer schemes and their impact on
Mars meteorology", available online at 
\verb+http://www-mars.lmd.jussieu.fr/WP2011/wp13.1.3d.pdf+
\end{itemize}
\nocite{Hour:93}
 
\subsection{Effects of subgrid orography and gravity waves}
 (\verb+ calldrag_noro+ ,  \verb+ drag_noro+ )

See {\it Forget et al. } [1999] and {\it Lott and Miller} [1997]
\nocite{Lott:97}

\section{Surface thermal conduction}
 (\verb+soil+)

The numerical scheme is described in section 2 of technical note
WP11.1 of ESA contract
Ref~ESA 11369/95/NL/JG(SC) "Improvement of the high latitude
processes in the Mars Global Climate Model", available online at
\verb+http://www-mars.lmd.jussieu.fr/WP2008/Polar_processes.pdf+
 
\section{CO$_2$ Condensation}

\begin{itemize}
\item In {\it Forget et al.} [1998] (article published in Icarus):
 \begin{itemize}
  \item Numerical method for calculating the condensation and sublimation levels
at the surface and in the atmosphere (\verb+ newcondens+)
 explained in the appendix.
  \item Description of the numerical scheme for calculating the evolution of CO$_2$
snow emissivity (\verb+co2snow+) explained in section 4.1
  \end{itemize}
\nocite{Forg:98}
\item Noncondensable gaz treatment: see {\it Forget et al.} [2008],
available online at
\verb+http://www.lpi.usra.edu/meetings/modeling2008/pdf/9106.pdf+
\item Inclusion of sub-surface water ice table thermal effect, varying albedo
of polar caps and tuning of the CO2 cycle are descibed in technical note
WP13.1.3e of ESA contract
Ref~ESA 11369/95/NL/JG(SC) "New Mars Global Climate Model:
e) Improved CO2 cycle and seasonal pressure variations", available online at
\verb+http://www-mars.lmd.jussieu.fr/WP2011/wp13.1.3e.pdf+
\end{itemize}


\section{Tracer transport and sources} 
\begin{itemize}
\item ``Van-Leer'' transport scheme used in the dynamical part
(\verb+ tracvl+ and  \verb+ vlsplt+ in the dynamical part):
{\it Hourdin and Armengaud} [1999] \nocite{Hour:99}

\item Transport by turbulent diffusion  (in \verb+ vdifc+), convection
(in  \verb+ convadj+), sedimentation  (\verb+ sedim+),
dust lifting by winds (\verb+ dustlift+) :
see note ``Preliminary design of dust lifting and transport in the Model''
(ESA contract, Work Package 4, 1998, available on the web). 

\item Dust transport by the ``Mass mixing ratio /
Number mixing ratio'' method for grain size evolution: see article by {\it
Madeleine et al.} [2011]
\nocite{Made:11}

%\item Simplified water cycle (source in {\tt vdifc}, {\tt
%watercloud}) : and also see the Maitrise study by Delphine Nobileau, LMD, 2000.
\item {\bf Watercycle}, see {\it Montmessin et al.} [2004]
and technical note
WP13.1.3c of ESA contract
Ref~ESA 11369/95/NL/JG(SC) "New Mars Climate Model: c) Inclusion of cloud
microphysics, dust scavenging and improvement of the water cycle",
available online at
\verb+http://www-mars.lmd.jussieu.fr/WP2011/wp13.1.3c.pdf+
\nocite{Mont:04jgr}

\item Radiative effect of clouds: see technical note
WP13.1.3b of ESA contract Ref~ESA 11369/95/NL/JG(SC)
"New Mars Climate Model: b) Radiative effects of water
ice clouds and impact on temperatures", available online at
\verb+http://www-mars.lmd.jussieu.fr/WP2011/wp13.1.3b.pdf+

%\item Chemistry, thermosphere, clouds: currently being published.
\item {\bf Chemistry}, see {\it Lef\`evre et al.} [2004]
and {\it Lef\`evre et al. [2008]}
\nocite{Lefe:04,Lefe:08}
\end{itemize}

\section{Thermosphere}
\begin{itemize}
\item A general description of the model is given in
{\it Gonz{\'a}lez-Galindo et al.} [2009]
\item Details on photochemistry and EUV radiative transfer can be found in
{\it Angelats i Coll et al.} [2005] and
{\it Gonz{\'a}lez-Galindo et al.} [2005]
\end{itemize}
\nocite{Gonz:09a,Gonz:09b,Gonz:05,Ange:05}
