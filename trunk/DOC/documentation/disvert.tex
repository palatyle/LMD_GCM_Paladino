\documentclass[a4paper,10pt]{article}
%\usepackage{graphicx}
\usepackage{natbib}  % si appel � bibtex
%\usepackage[francais]{babel}
%\usepackage[latin1]{inputenc}  % accents directs (�...), avec babel
%\usepackage{rotating}

\setlength{\hoffset}{-1.in}
\setlength{\oddsidemargin}{3.cm}
\setlength{\textwidth}{15.cm}
\setlength{\marginparsep}{0.mm}
\setlength{\marginparwidth}{0.mm}

\setlength{\voffset}{-1.in}
\setlength{\topmargin}{0.mm}
\setlength{\headheight}{0.mm}
\setlength{\headsep}{30.mm}
\setlength{\textheight}{24.cm}
\setlength{\footskip}{1.cm}

\setlength{\parindent}{0.mm}
\setlength{\parskip}{1 em}
\newcommand{\ten}[1]{$\times 10^{#1}$~} 
\renewcommand{\baselinestretch}{1.}

\begin{document}
\pagestyle{plain}

\begin{center}
{\bf \LARGE 
Documentation for LMDZ, Planets version

\vspace{1cm}
\Large
The vertical discretization
}

\vspace{1cm}
S\'ebastien Lebonnois, Ehouarn Millour

\vspace{1cm}
Latest version: \today
\end{center}


\section{Theoretical aspects}

The position of the layers: 
\begin{itemize}
\item pressure limit between two layers, 
\item pressure within the layers
\end{itemize}

The Exner function: 
$pk = C_p \times (p/preff)^\kappa$.
It corresponds to the pressure levels within the layers.
Used for the computation of the potential temperature.
For the Earth, we use a specific scheme that computes these positions so that 
it maintains a condition of proportionality between total,
internal and potential energy (cf. a note from F. Hourdin). 

\section{Pratical aspects in the code}

\begin{itemize}
\item \textsf{disvert\_[no]terre.F[90]}: 
position of the interface pressure levels from an input file
(several possibilities). 
Definition of ap, bp and presnivs.
In the planetary version, definition of aps and bps.

This is done only once, called at the beginning from \textsf{iniconst.F}.

In the Earth version the vertical coordinates are hybrid (sigma-pressure),
and generated automaticaly (or generated from parameters read from file
\textsf{sigma.def}, if that file is present in the directory where the
gcm is run).

In the planetary version, the vertical coordinates can be hybrid (default
behavior) or sigma (set using parameter "hybrid" in \textsf{run.def}; true
implies hybrid coordinate, false implies sigma coordinate). the distribution
of model levels is set from file \textsf{esasig.def} or \textsf{z2sig.def},
depending on which is present (in the directory where the gcm is run).
The first line of the  \textsf{z2sig.def} file should give the value of the
reference atmospheric scale height (in km), followed by the (rough estimate)
of the altitude (in km) of the atmospheric level (one per line of the file).

For planetary applications, the usual way to go is to use \textsf{disvert\_noterre.F} together with \textsf{z2sig.def}.
The sigma levels are computed as: 
\begin{itemize}
\item $H$ is the first value in \textsf{z2sig.def}, $zsig$ are the following values in \textsf{z2sig.def}
\item $\sigma(1)=1$
\item l=2 to llm: $\sigma(l)= 0.5 \times (\exp(-zsig(l)/H)+\exp(-zsig(l-1)/H))$
\item $\sigma(llm+1)=0$
\end{itemize}
The $ap$ and $bp$ values, defining the interface pressure levels, are then computed using these sigma values, the reference pressure $preff$, and the transition pressure $pa$. The values of $preff$ and $pa$ are read in the \textsf{start.nc} file, in the control array.

\item Interface pressures: 
computed in \textsf{caldyn0.F, caldyn.F, integrd.F, leapfrog.F}
through the \textsf{pression.F} routine.

\item Exner function (and therefore pressure within the layers): 
computed at three different places in \textsf{leapfrog.F} through the 
\textsf{exner\_[hyb/milieu].F} routine. 
For the Earth, we use \textsf{exner\_hyb.F}, that computes the positions in a
specific way to maintain a condition of proportionality between total,
internal and potential energy (cf. a note from F. Hourdin). 
For other planets, we use \textsf{exner\_milieu.F}, that computes the positions
of these pressure levels exactly in the middle of each layer. 
Though this fails to maintain the previous condition, there is no evidence of 
any significant influence on the results, and it makes it a lot easier to 
define correctly the level positions with the input file.
\end{itemize}

%\begin{thebibliography}{2}
%\providecommand{\natexlab}[1]{#1}
%\expandafter\ifx\csname urlstyle\endcsname\relax
%  \providecommand{\doi}[1]{doi:\discretionary{}{}{}#1}\else
%  \providecommand{\doi}{doi:\discretionary{}{}{}\begingroup
%  \urlstyle{rm}\Url}\fi

%\end{thebibliography}

\end{document}
