\documentclass[a4paper,10pt]{article}
%\usepackage{graphicx}
\usepackage{natbib}  % si appel � bibtex
%\usepackage[francais]{babel}
%\usepackage[latin1]{inputenc}  % accents directs (�...), avec babel
%\usepackage{rotating}

\setlength{\hoffset}{-1.in}
\setlength{\oddsidemargin}{3.cm}
\setlength{\textwidth}{15.cm}
\setlength{\marginparsep}{0.mm}
\setlength{\marginparwidth}{0.mm}

\setlength{\voffset}{-1.in}
\setlength{\topmargin}{0.mm}
\setlength{\headheight}{0.mm}
\setlength{\headsep}{30.mm}
\setlength{\textheight}{24.cm}
\setlength{\footskip}{1.cm}

\setlength{\parindent}{0.mm}
\setlength{\parskip}{1 em}
\newcommand{\ten}[1]{$\times 10^{#1}$~} 
\renewcommand{\baselinestretch}{1.}

\begin{document}
\pagestyle{plain}

\begin{center}
{\bf \LARGE 
Documentation for LMDZ, Planets version

\vspace{1cm}
\Large
Running with only one column: \\
rcm1d in Venus and Titan physics
}

\vspace{1cm}
S\'ebastien Lebonnois

\vspace{1cm}
Latest version: \today
\end{center}


\section{The \textsf{rcm1d} tool}

The file \textsf{rcm1d.F} is located in the \textsf{phy$<$planet$>$} directory.
For the moment, the tool described here is available for Venus and Titan, though a similar tool exists for Mars and the Generic model.

The goal of this tool is to initialize the model the same way \textsf{gcm.F} does on 3D, but only on a single column, so that the physics may be tested without any dynamics.

It can be compiled (sequential only) with a command like (e.g. for 50 layers):

\textsf{makelmdz -p venus -d 50 rcm1d}

It requires the files \textsf{rcm1d.def}, \textsf{physiq.def}
and a file describing the vertical layers (\textsf{z2sig.def}).
The 3D \textsf{run.def} is also needed though only to access \textsf{physiq.def} (the other parameters specific to \textsf{run.def} are not read.

{\bf Beware:}
The file \textsf{traceur.def} may or may not be present. 
For the moment, tracers are initialized to 0. 
If you plan to use them, there will be need for modifications (in \textsf{rcm1d.F}).


Using these files and an initial temperature vertical profile defined in the \textsf{profile.F} routine (with options available through \textsf{rcm1d.def}), the model is initialized and a first \textsf{startphy.nc} file is written, to be read again at the first call of \textsf{physiq}.

\section{Specific \textsf{rcm1d.def} file}
 
This file is read by \textsf{rcm1d} during initialization.
It is very simple and reads values at the beginning of each lines (different from \textsf{gcm.def}).
It can be found in the \textsf{deftank} directory. 
Each line has a comment explaining what the parameter is. 
The number of timestep per day (third line) is the number of calls to the physics per day, since no dynamics is involved here.

It also includes parameters for the \textsf{profile.F} definition.

\section{Outputs}

It writes the same \textsf{hist*.nc} as the regular GCM.
It also writes a \textsf{profile.new} file containing the vertical profiles of altitude, temperature and stability at the end of the run, and the usual \textsf{restartphy.nc} though it will not be used for a restart, since \textsf{rcm1d} starts with the same initial state everytime it is launched.

\end{document}
