\documentclass[a4paper,10pt]{article}
%\usepackage{graphicx}
\usepackage{natbib}  % si appel � bibtex
%\usepackage[francais]{babel}
%\usepackage[latin1]{inputenc}  % accents directs (�...), avec babel
%\usepackage{rotating}

\setlength{\hoffset}{-1.in}
\setlength{\oddsidemargin}{3.cm}
\setlength{\textwidth}{15.cm}
\setlength{\marginparsep}{0.mm}
\setlength{\marginparwidth}{0.mm}

\setlength{\voffset}{-1.in}
\setlength{\topmargin}{0.mm}
\setlength{\headheight}{0.mm}
\setlength{\headsep}{30.mm}
\setlength{\textheight}{24.cm}
\setlength{\footskip}{1.cm}

\setlength{\parindent}{0.mm}
\setlength{\parskip}{1 em}
\newcommand{\ten}[1]{$\times 10^{#1}$~} 
\renewcommand{\baselinestretch}{1.}

\begin{document}
\pagestyle{plain}

\begin{center}
{\bf \LARGE 
Documentation for LMDZ, Planets version

\vspace{1cm}
\Large
The horizontal dissipation
}

\vspace{1cm}
S\'ebastien Lebonnois

\vspace{1cm}
Latest version: \today
\end{center}

\section{Theoretical aspects}

To be written

\section{Pratical aspects in the code}

The horizontal dissipation parameters are chosen in \textsf{gcm.def}.

{\bf Parameters related to the operators}

\begin{itemize}
\item \textsf{idissip}: timestep for dissipation. 
Should be equal to \textsf{iperiod}. 
\item \textsf{lstardis}: boolean that indicates 
whether to use a star operator (or not). Usually set to {\em True}.
\item \textsf{nitergdiv}: number of iterations for the {\em gradiv} operator
\item \textsf{nitergrot}: number of iterations for the {\em nxgradrot} operator
\item \textsf{niterh}:  number of iterations for the {\em divgrad} operator
\end{itemize}
 
{\bf Parameters related to timescales}

\begin{itemize}
\item \textsf{tetagdiv}: time scale (in s) for the {\em gradiv} operator. 
It corresponds to the attenuation of the smallest wavelengths for u and v 
perturbations.
\item \textsf{tetagrot}: time scale (in s) for the {\em nxgradrot} operator. 
It corresponds to the attenuation of the smallest wavelengths for u and v 
perturbations.
\item \textsf{tetatemp}: time scale (in s) for the {\em divgrad} operator. 
It corresponds to the attenuation of the smallest wavelengths for h 
perturbations.
\end{itemize}
These timescales are the one for the deep atmosphere. 
However, they are modified by some factors as the pressure decreases. 
For these factors, two steps are implemented: \textsf{dissip\_fac\_mid} and
\textsf{dissip\_fac\_up} (also chosen in \textsf{gcm.def}). 
The first step is always applied, the second is
applied only when \textsf{ok\_strato} is set to {\em True}.
Timescales are divided by the factor $f$ computed as detailed below. 

For \textsf{dissip\_fac\_mid} (which is usually equal to 2.), transition is 
computed with:
\[
         x = 1. - \frac{{\rm preff}}{p}
\]
\[
         f_1 = {\rm dissip\_fac\_mid} -
           \frac{{\rm dissip\_fac\_mid}-1.}{1.+x^2}
\]

When applied (\textsf{ok\_strato} set to {\em True}), the transition from
\textsf{dissip\_fac\_mid} to \textsf{dissip\_fac\_up} is done with a 
$\tanh$ function, using further parameters:
\begin{itemize}
\item \textsf{dissip\_deltaz}: altitude range (in km) for the transition.
\item \textsf{dissip\_hdelta}: scale height (in km) at the altitude of the transition.
\item \textsf{dissip\_pupstart}: pressure (in Pa) corresponding to the bottom 
of the transition region.
\end{itemize}
The function used is:
\[
         f_2 = \left[ 1+
     \left( \frac{{\rm dissip\_fac\_up}}{{\rm dissip\_fac\_mid}} - 1. \right)
     \times \left( 1 - \frac{1 + \tanh X}{2} \right) \right]
\]
with
\[
    X = \frac{6.\times {\rm dissip\_hdelta}}{{\rm dissip\_deltaz}}
      \log \frac{p}{{\rm dissip\_pupstart} \times 
     \exp \left( \frac{-{\rm dissip\_deltaz}}{2 {\rm dissip\_hdelta}} \right)}
%      = \frac{6.\times {\rm dissip\_hdelta}}{{\rm dissip\_deltaz}}
%      \log \frac{p}{{\rm dissip\_pupstart}} + 3.
\]

{\bf A last parameter\dots}

There is one last coefficient in \textsf{gcm.def} related to dissipation: 
\textsf{coefdis}. 
It is used in \textsf{inigeom.F} to compute \textsf{gamdi\_gdiv, gamdi\_grot, 
gamdi\_h}. Usually put to 0... Should be explained here...




%\begin{thebibliography}{2}
%\providecommand{\natexlab}[1]{#1}
%\expandafter\ifx\csname urlstyle\endcsname\relax
%  \providecommand{\doi}[1]{doi:\discretionary{}{}{}#1}\else
%  \providecommand{\doi}{doi:\discretionary{}{}{}\begingroup
%  \urlstyle{rm}\Url}\fi

%\end{thebibliography}

\end{document}
