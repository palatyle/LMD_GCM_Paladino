\documentclass[a4paper,10pt]{article}
%\usepackage{graphicx}
\usepackage{natbib}  % si appel � bibtex
%\usepackage[francais]{babel}
%\usepackage[latin1]{inputenc}  % accents directs (�...), avec babel
%\usepackage{rotating}

\setlength{\hoffset}{-1.in}
\setlength{\oddsidemargin}{3.cm}
\setlength{\textwidth}{15.cm}
\setlength{\marginparsep}{0.mm}
\setlength{\marginparwidth}{0.mm}

\setlength{\voffset}{-1.in}
\setlength{\topmargin}{0.mm}
\setlength{\headheight}{0.mm}
\setlength{\headsep}{30.mm}
\setlength{\textheight}{24.cm}
\setlength{\footskip}{1.cm}

\setlength{\parindent}{0.mm}
\setlength{\parskip}{1 em}
\newcommand{\ten}[1]{$\times 10^{#1}$~} 
\renewcommand{\baselinestretch}{1.}

\begin{document}
\pagestyle{plain}

\begin{center}
{\bf \LARGE 
Documentation for LMDZ, Planets version

\vspace{1cm}
\Large
The upper boundary sponge layer
}

\vspace{1cm}
S\'ebastien Lebonnois , Ehouarn Millour

\vspace{1cm}
Latest version: \today
\end{center}

\section{Theoretical aspects}
Because of the inevitable numerical boundary at the top of the model,
upward travelling waves are found to non-physically reflect down into the
atmosphere.
A common remedy to this unwanted behaviour is to apply a sponge layer near
the top of the model in order to quench these waves and avoid significant
reflection thereof.
In practice such quenching is done by adding a dissipative term which forces
a relaxation of potential temperature and/or winds of the form:
\[
 A(t)=A_m+A_0 \exp(-\lambda t )
\]
Where $A_m$ is the value towards which $A$ is to asymptotically relax, and
$\lambda$ is the inverse of the characteristic relaxation time scale.
As there is no obvious value of $A_m$ towards which to relax, in practice
it is often chosen to be either the zonal average of $A$ (evaluated at time $t$,
i.e. conveniently ignoring that $A_m$ then is in fact not time-independent),
or zero (at least for winds, since this would have little physical meaning for
potential temperature).

\section{Pratical aspects in the code}

%The sponge layer is applied at the upper boundary when the \textsf{ok\_strato}
%flag is set to {\em True} in \textsf{gcm.def} 
%(this parameter also controls the application of a second step in the 
%determination of vertical variation of coefficients for
%the horizontal dissipation, see \textsf{inidissip.F} and
%\textsf{disspi\_horiz.pdf} document).

The tendencies for the upper boundary sponge layer are computed separately in
the \textsf{top\_bound.F} routine (called from \textsf{leapfrog.F}) and
added in place. 
The resulting sponge tendency \textsf{dutop}, in m/s, is also given as an output for
diagnostics.

Three parameters may be adjusted in the \textsf{gcm.def} file: 
\begin{itemize}
\item \textsf{iflag\_top\_bound}: selects the affected layers. 
  \begin{itemize}
  \item 1: only the top 4 layers are affected. In this case, the damping rate 
  is divided by 2 in the second layer, 4 in the third and 8 in the fourth. 
  \item 2: layers with pressure lower than 100 times the top pressure. 
  In this case, the damping rate depends linearly on the pressure.
  \end{itemize}
\item \textsf{mode\_top\_bound}: selects how the fields are affected.
  \begin{itemize}
  \item 0: No sponge layer is applied.
  \item 1: Zonal and meridional winds are damped to zero.
  \item 2: Zonal and meridional winds are damped to their zonally averaged value.
  \item 3: Temperature, zonal and meridional winds are damped to their zonally 
  averaged value.
  \end{itemize}
\item \textsf{tau\_top\_bound}: damping rate ($\lambda$ in equation above,
expressed in Hz) in the topmost layer.
\end{itemize}

%\begin{thebibliography}{2}
%\providecommand{\natexlab}[1]{#1}
%\expandafter\ifx\csname urlstyle\endcsname\relax
%  \providecommand{\doi}[1]{doi:\discretionary{}{}{}#1}\else
%  \providecommand{\doi}{doi:\discretionary{}{}{}\begingroup
%  \urlstyle{rm}\Url}\fi

%\end{thebibliography}

\end{document}
