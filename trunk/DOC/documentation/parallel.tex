\documentclass[a4paper,10pt]{article}
%\usepackage{graphicx}
\usepackage{natbib}  % si appel � bibtex
%\usepackage[francais]{babel}
%\usepackage[latin1]{inputenc}  % accents directs (�...), avec babel
%\usepackage{rotating}

\setlength{\hoffset}{-1.in}
\setlength{\oddsidemargin}{3.cm}
\setlength{\textwidth}{15.cm}
\setlength{\marginparsep}{0.mm}
\setlength{\marginparwidth}{0.mm}

\setlength{\voffset}{-1.in}
\setlength{\topmargin}{0.mm}
\setlength{\headheight}{0.mm}
\setlength{\headsep}{30.mm}
\setlength{\textheight}{24.cm}
\setlength{\footskip}{1.cm}

\setlength{\parindent}{0.mm}
\setlength{\parskip}{1 em}
\newcommand{\ten}[1]{$\times 10^{#1}$~} 
\renewcommand{\baselinestretch}{1.}

\begin{document}
\pagestyle{plain}

\begin{center}
{\bf \LARGE 
Documentation for LMDZ, Planets version

\vspace{1cm}
\Large
Running the GCM in parallel using MPI
-- Venus
}

\vspace{1cm}
S\'ebastien Lebonnois

\vspace{1cm}
Latest version: \today
\end{center}


\section{Compilation}

To compile the GCM for parallel runs using MPI, you need to find the MPI compilor (\textsf{mpif90}) you want to use on your machine. With that knowledge, you can build you own \textsf{arch-$<$your\_architecture$>$.fcm} file in the \textsf{LMDZ.COMMON/arch/} directory. 
You can find inspiration, for example, on the \textsf{arch-GNOMEp.fcm} (for the {\em gnome} computation server of UPMC), which uses \textsf{ifort}. 
For the LMD local computation machines (e.g. {\em levan}), you can use \textsf{arch-linux-64bit-para.fcm}.

You also need to have \textsf{netcdf} and \textsf{ioipsl} compiled using the same compilor and main options. The paths to these libraries (and includes) must be written in the \textsf{arch-$<$your\_architecture$>$.path} file.

An example of command line to compile the Venus GCM using \textsf{makelmdz} is then:

\textsf{makelmdz -arch $<$your\_architecture$>$ -parallel mpi -d $<$nlon$>$x$<$nlat$>$x$<$nlev$>$ -p venus gcm}

\section{Run}
 
To run the simulation, you have to use the \textsf{mpirun} launcher corresponding to your \textsf{mpif90} compilor.

The command line is:

\textsf{mpirun -n $<$number\_of\_procs$>$ gcm.e}

\section{Outputs}

Each of the processors used during the run will write its own portion of the \textsf{hist$<$mth/day/ins$>$.nc} files. To gather these portions back into one single file, there is a tool located in the ioipsl directory that you built from the SVN instructions. 

This tool is called \textsf{rebuild} and is found in the \textsf{ioipsl/modipsl/bin/} directory. 

To use it, the command line is:

\textsf{rebuild -f -o $<$name\_of\_final\_file$>$.nc hist$<$mth/day/ins$>$\_*.nc}

\end{document}
