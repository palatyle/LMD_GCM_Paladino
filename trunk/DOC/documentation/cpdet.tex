\documentclass[a4paper,10pt]{article}
%\usepackage{graphicx}
\usepackage{natbib}  % si appel � bibtex
%\usepackage[francais]{babel}
%\usepackage[latin1]{inputenc}  % accents directs (�...), avec babel
%\usepackage{rotating}

\setlength{\hoffset}{-1.in}
\setlength{\oddsidemargin}{3.cm}
\setlength{\textwidth}{15.cm}
\setlength{\marginparsep}{0.mm}
\setlength{\marginparwidth}{0.mm}

\setlength{\voffset}{-1.in}
\setlength{\topmargin}{0.mm}
\setlength{\headheight}{0.mm}
\setlength{\headsep}{30.mm}
\setlength{\textheight}{24.cm}
\setlength{\footskip}{1.cm}

\setlength{\parindent}{0.mm}
\setlength{\parskip}{0 ex}
\newcommand{\ten}[1]{$\times 10^{#1}$~} 
\renewcommand{\baselinestretch}{1.}

\begin{document}
\pagestyle{plain}

\begin{center}
{\bf \LARGE 
Documentation for LMDZ, Planets version

\vspace{1cm}
\Large
The Cp(T) dependency
}

\vspace{1cm}
S\'ebastien Lebonnois

\vspace{1cm}
Latest version: \today
\end{center}

\section{Theoretical aspects}

Taken from \cite{lebonnois10}.

The standard version of the LMDZ dynamical core uses a single value for the 
specific heat $C_p$, but $C_p$ varies in the atmosphere of Venus from 
738~J/kg/K at 100~km altitude to 1181~J/kg/K near the surface 
\cite[values taken from the Venus International Reference Atmosphere,][]{seiff85}. 
This variation of $C_p$ with temperature needs to be taken into account, 
in order to get realistic adiabatic lapse rates in the whole atmosphere. 
We use an analytic approximation for this temperature dependence, 
that yields values very close to the VIRA profile for $C_p$ 
(within 4\% for temperatures below 200~K, below 1\% everywhere else):

\begin{equation} 
C_p(T) = C_{p_0} \times \left(\frac{T}{T_0}\right)^{\nu}, 
\label{eq:cpdet}
\end{equation} 
with $C_{p_0} = 1000$~J/kg/K, $T_0 = 460$~K, and $\nu = 0.35$.

In the LMDZ model, the potential temperature is one of the fundamental 
pronostic variable in the equations of energy conservation. 
The potential temperature $\theta$ is the temperature an air parcel initially at
temperature $T$ and pressure $p$ would reach when brought to a reference
pressure $p_{\rm ref}$ (typically, the surface pressure) adiabatically:

\begin{equation}
\int_\theta^T{C_p \frac{dT}{T}} = \int_{p_{\rm ref}}^p{R \frac{dp}{p}}
\label{eq:tpot}
\end{equation}
($R$ is the atmospheric gas constant). Equation~\ref{eq:tpot} is
obtained from the first principle of thermodynamics when a parcel of air at
pressure $p$ and temperature $T$ is brought adiabatically to the reference
pressure $p_{\rm ref}$. 

When $C_p$ is taken as constant with temperature, Eq.~\ref{eq:tpot} yields 
the classical expression of potential temperature 
$\theta = T \times (p_{\rm ref}/p)^\kappa$, with $\kappa = R/C_p$. 
However, when $C_p$ depends on temperature, this expression is no longer valid.
 
To keep modifications of the dynamical core to the minimum, we have kept 
potential temperature as a key variable in the dynamics. Thus, we have 
calculated the new expression of potential temperature under these conditions.
Introducing the expression of $C_p(T)$ (Eq.~\ref{eq:cpdet}) in Eq.~\ref{eq:tpot} yields

\begin{equation}
\int_\theta^T{\frac{T^{\nu-1}}{T_0^\nu}dT}
 = \frac{R}{C_{p_0}} \ln \frac{p}{p_{\rm ref}},
\end{equation}
then (for $\nu \neq 0$)

\begin{equation}
\frac{1}{\nu T_0^\nu} \left( T^\nu - \theta^\nu \right) 
 = \ln \left( \frac{p}{p_{\rm ref}} \right)^{\kappa_0},
\end{equation}
where $\kappa_0 = \frac{R}{C_{p_0}}$. This yields the new expression for the
potential temperature:

\begin{equation}
\theta^\nu = T^\nu + \nu T_0^\nu \ln \left( \frac{p_{\rm ref}}{p} \right)^{\kappa_0}.
\label{eq:newtpot}
\end{equation}

The adjustments done in the dynamical core enable us to run the GCM with any
formulation of $C_p$ and the corresponding potential temperature based on 
Eq.~\ref{eq:tpot}.

\section{Pratical aspects in the code}

A specific file has been added to the dynamical core, \textsf{cpdet.F}, which includes 
all the needed routines to take the $C_p(T)$ possibility into account.
These routines take advantage of the keyword \textsf{planet\_type} to 
implement different expressions of $C_p(T)$. These routines are:
\begin{itemize}
\item \textsf{ini\_cpdet}: initializes the parameters $\nu$ (\textsf{nu\_venus})
and $T_0$ (\textsf{t0\_venus}) to either the Venus values, or zero. 
It is called just once at the begining of the main routine.
These parameters are declared in \textsf{comconst.h}.
\item \textsf{cpdet}: function, that computes $C_p$ for a given $T$. 
For other planets than Venus, it is just \textsf{cpdet(T)=cpp}.
\item \textsf{t2tpot}: converts a temperature vector to a potential temperature
vector.
\item \textsf{tpot2t}: converts a potential temperature vector to a temperature
vector.
\end{itemize}

In the routines, instead of using the constants \textsf{cpp} (dynamical core) 
or \textsf{RCPD} (physical module), we need to use the 
function \textsf{cpdet(T)} where T is the temperature at a given point.

Since the potential temperature is the variable used in the dynamical core,
the temperature is scarcely used. But in some places, it was computed directly 
from the potential temperature and the Exner function 
${\rm pk} = {\rm cpp}\times(\frac{\rm play}{\rm pref})^{\rm kappa}$.
The routine \textsf{tpot2t} is now used to compute the temperature when needed,
and the variable \textsf{tsurpk}, which is the temperature over the Exner 
function can be used as an argument instead of the former potential 
temperature (for \textsf{sortvarc0.F}, \textsf{sortvarc.F}, \textsf{dudv2.F}
and \textsf{geopot.F}).
This affects the routines (in \textsf{dyn3d}):
\begin{itemize}
\item \textsf{caldyn0.F}
\item \textsf{caldyn.F}
\item \textsf{calfis.F}
\item \textsf{diagedyn.F}
\item \textsf{leapfrog.F}
\item \textsf{vlspltqs.F}
\end{itemize}

\begin{thebibliography}{2}
\providecommand{\natexlab}[1]{#1}
\expandafter\ifx\csname urlstyle\endcsname\relax
  \providecommand{\doi}[1]{doi:\discretionary{}{}{}#1}\else
  \providecommand{\doi}{doi:\discretionary{}{}{}\begingroup
  \urlstyle{rm}\Url}\fi

\bibitem[{\textit{Lebonnois et~al.}(2010)\textit{Lebonnois, Hourdin, Eymet,
  Crespin, Fournier, and Forget}}]{lebonnois10}
Lebonnois, S., F.~Hourdin, V.~Eymet, A.~Crespin, R.~Fournier, and F.~Forget
  (2010), {Superrotation of Venus' atmosphere analysed with a full General
  Circulation Model}, \textit{J. Geophys. Res.}, \textit{115}, E06006,
  \doi{10.1029/2009JE003458}.

\bibitem[{\textit{Seiff et~al.}(1985)\textit{Seiff, Schofield, Kliore
  et~al.}}]{seiff85}
Seiff, A., J.~T. Schofield, A.~J. Kliore, et~al. (1985), {Model of the
  structure of the atmosphere of Venus from surface to 100 km altitude},
  \textit{Adv. Space Res.}, \textit{5}(11), 3--58.

\end{thebibliography}

\end{document}
